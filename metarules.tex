\documentclass[tom-drew]{subfiles}
\begin{document}
	
	\chapter{Definitions \& Meta Rules}
	
	This section is to help define ambiguous situations. Historically we have struggled on occasion with bidding situations (mostly competitive) which arise due to different baseline assumptions.  We will attempt to capture some examples of problems we've had, items we have hashed out or examples of confusion at the table.
	
	These are Tom's default rules.  Any or all of the following can be overridden with specific exceptions carved out in the notes, but when there is no other specific agreement these rules are expected.
	
	\begin{description}
		\item[Bal] \shape{4333}, \shape{4432} \& \shape{5332} are always considered balanced. \shape{5422} \& \shape{6322} are often treated as balanced, but can be considered unbalanced. \shape{7222}, \shape{4441} \& \shape{5431} with high honor singletons are generally treated as unbalanced but can be treated as balanced if needed.
		\item[GF] GF auctions cannot end in 4 of a minor, even after an attempt to get to 3NT has failed.
		\item[OKC] 4\clubsuit/\diamondsuit is Optional Key Card (aka Qwood) when (a) the bid is forcing and (b) the strain is viable.
		\item[2NT] 2NT in competition defaults to scrambling. Good/Bad when explicitly defined.
		\item[1NT] Over our 1NT overcalls, we generally play systems on. This includes Lebensohl agreements should the opps continue to bid.
		\item[WJO] Over opp's weak jump overcalls to the 2 level, we do not treat the bid as if they opened a weak 2. In response to auctions such as \cl1--(\sp2)--Dbl, 2NT is natural and NF
		\item[Dbls] Low level doubles are card showing, generally for takeout. If the auction is ``dead'' then it is for penalty. (A dead auction is one which both partners have limited their hands and attempted to play a partscore and the opponents balance.)
		\item[1x Resp.] Responder is expected to bid with any hand containing:
		\begin{enumerate}
			\item An Ace
			\item A King and 5 HCP
			\item 6 HCP
		\end{enumerate}
		\item[Overcalls] 1 level overcalls are typically 5 but allowed to be 4 in a major with a hand unsuitable for a takeout double. Our (non-jump) 2 level overcalls are expected to be 6 cards when a minor, although 5 is possible in rare circumstances with strong suits. (\sp1)--\he2 can be on a modest 5 card suit.
	\end{description} 
	
\end{document}